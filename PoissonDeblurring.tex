\documentclass{beamer}
\usetheme{Warsaw}
%\definecolor{mycolor}{rgb}{.125,0.5,0.25}
%\usecolortheme[named=mycolor]{structure}
% remove the navigation bar
\beamertemplatenavigationsymbolsempty
% set the numbering to fraction 1/n
\setbeamertemplate{frame numbering}[fraction]
% add the frame count to the footer 
\setbeamertemplate{footline}[frame number]

\title[Poisson Deblurring]{Poisson Image Deblurring}
\author{Sek Cheong, Das Deepan} 
\institute{University of Wisconsin, Madison}
\date{March 15 2019}

\begin{document}

\begin{frame}
\titlepage
\end{frame}

\begin{frame}[t]{Introduction}
Some introuction text

\end{frame}

\begin{frame}[t]{Review K-SVD}{Definitions}
Principles of K-SVD 
    \begin{itemize}   
        \item Want to construct the composite image, $I(x,y)$, which should agree with $T(x,y)$ and look like $S(x,y)$    
        \item $I(x,y)$ should exactly agree with $T(x,y)$ 
        \item $I(x,y)$ ``look like'' $S(x,y)$ inside $\Omega$
        \item If we place directly $S$ over $T$ and smooth over the edges, the result maybe unacceptable, due to color mismatch
    \end{itemize}
\end{frame}

\begin{frame}[t]{Review K-SVD}{Mathematical Background}
K-SVD Mathematical background

A fundamental equation of {\bf Calculus of Variations} states that, if $J$ is defined by an integral of the form:
\[
    J= \int F(x,f, f_x)dx
\]
Then $J$ has a stationary value if the following differential equation is satisfied:
\[
    \frac{\partial F}{\partial f} - \frac{d}{dx}\frac{\partial F}{\partial f_x}=0
\]
\end{frame}

\begin{frame}[t]{Proposed Model}
Mathematical model details
\end{frame}

\begin{frame}[t]{Experiment Results}
some exprement results
\end{frame}

\begin{frame}[t]{Conclusion}
Conclusion
\end{frame}

\end{document}